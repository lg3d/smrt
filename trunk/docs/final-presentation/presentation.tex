\documentclass[style=smrt,mode=present,paper=screen]{powerdot}
% \documentclass[style=smrt,mode=handout,paper=letterpaper]{powerdot}

\usepackage{setspace}
\usepackage{listings}

\lstnewenvironment{code}{%
        \lstset{basicstyle=\tiny\ttfamily}
}{}

%\onehalfspacing

\title{\huge{\textbf{smrt Final Demonstration}} \\ \small{A 3D Media Center}}
\author {
	Cory Maccarrone
	\and
	Daniel Seikaly
	\and
	Evan Sheehan
	\and
	David Trowbridge
}
\date{May 2, 2006}
\begin{document}
\maketitle

\presenter{Daniel Seikaly}

% Overview slide.  Sort of a table of contents.
\begin{slide}[toc=,bm=]{Presentation Overview}
{\footnotesize \tableofcontents[content=all]}
\end{slide}

%\doublespacing

% About the class
\section[slide=false]{The Class}
\begin{slide}{Senior Project Class}
\begin{itemize}
\item Year-long course
\item Required for all computer science sutdents at CU
\item Taught by Bruce Sanders
\item Offered every year since its inception in 1987
\item Since then 1140 students have finished 273 projects
\item \textit{smrt} is one of 15 projects involving 68 students
\item Other projects vary from card games to 3D image conversion tools
\item Its purpose is to give students ``real world'' experience
\end{itemize}
\end{slide}

% The Problem
\section[slide=false]{The Problem}
\begin{slide}[toc=,bm=]{Problem: Representation}
\begin{itemize}
\item Current media center software designed in 2D \\
\item Can we benefit from a third dimension? \\
\begin{itemize}
	\item More efficient representation of data? \\
	\item More intuitive user interface? \\
	\item Eye candy
\end{itemize}
\end{itemize}
\end{slide}

% The "Solution"
\section[slide=false]{The Solution}
\begin{slide}[toc=,bm=]{The Solution: smrt}
\begin{itemize}
\item Media center application based on Sun's Looking Glass 3D
\begin{itemize}
	\item Framework for 3D interfaces
	\item Input device events
	\item Embedding external programs into the 3D environment
\end{itemize}
\item Provides TiVO-like functionality in a 3D environment
\item Capabilities for navigating through vast libraries of media
\begin{itemize}
	\item Categories
	\item Filesystem navigation
\end{itemize}
\item Watch live TV, and play DVDs and music
\end{itemize}
\end{slide}

\presenter{Cory Maccarrone}

% Conceptual Diagram
\begin{slide}{Conceptual Diagram}
\begin{figure}[htb]
	\includegraphics[width=4in]{../lib/figures/conceptual_overview}
\end{figure}
\end{slide}

% Environmental Requirements
\begin{slide}{Environmental Requirements}
\begin{itemize}
\item Software Environment
\begin{itemize}
	\item Sun Java Development Kit version 5.0
	\item Latest version of Looking Glass 3D (currently 0.7.1)
\end{itemize}
\item Hardware Environment
\begin{itemize}
	\item Demonstrable on hardware supporting TV/HDTV output
	\item Input from remote control or keyboard
	\item Extensible to new and esoteric input devices
\end{itemize}
\end{itemize}
\end{slide}

% Functional Requirements
\begin{slide}{Functional Requirements}
\begin{itemize}
\item Display at multiple screen resolutions (NTSC and HDTV)
\item Provide mechanisms for displaying all available media
\begin{itemize}
	\item Local media (mp3s, etc.)
	\item DVDs, audio CDs
	\item Live TV (with schedules)
\end{itemize}
\item Interface must utilize the third dimension
\end{itemize}
\end{slide}

\presenter{David Trowbridge}

% Screenshots
\begin{slide}{Interface}
\begin{figure}[htb]
	\includegraphics[angle=-90,width=4in]{figures/interface}
\end{figure}
\end{slide}

% Demo
\begin{slide}{Demonstration}
\end{slide}

\presenter{Evan Sheehan}

% Architecture
\begin{slide}{Architecture}
\begin{itemize}
\item MediaCenter creates the StateController and ApplicationManager
\item StateController tracks the state with Context stack
\item ApplicationManager oversees external applications
\end{itemize}
\begin{figure}
	\includegraphics[angle=-90,width=4in]{figures/MediaCenter-uml}
\end{figure}
\end{slide}

\begin{slide}{Context}
\begin{itemize}
\item Context is the interface used to pass on events from the StateController
\item The context can be a menu or an application
\end{itemize}
\begin{figure}[htb]
	\includegraphics[angle=-90,width=4in]{figures/Context-uml}
\end{figure}
\end{slide}

\begin{slide}{Menus}
\begin{figure}[htb]
	\includegraphics[angle=-90,width=4in]{figures/Menu-uml}
\end{figure}
\end{slide}

\begin{slide}{Applications}
\begin{figure}[htb]
	\includegraphics[angle=-90,width=4in]{../lib/figures/ApplicationFactory-uml}
\end{figure}
\end{slide}

\begin{slide}{Menu files}
\begin{itemize}
        \item Menus are referenced by filename -- loading ``main'' will use ``main.menu''
	\item Uses Java's XML serialization
	\item File contains one Menu object and any number of Item objects
	\item Specialized menu types (such as FilesystemCity) can have special
	      configuration without upsetting other subsystems
\end{itemize}
\end{slide}

\begin{slide}[method=direct]{Example .menu}
\begin{code}
<java version="1.5.0" class="java.beans.XMLDecoder">
  <object class="org.jdesktop.lg3d.apps.smrt.menu.RingMenu"/>
  <object class="org.jdesktop.lg3d.apps.smrt.menu.IconItem">
    <void property="iconFilename">   <string>tv.png</string>   </void>
    <void property="label">          <string>Watch TV</string> </void>
    <void property="action">         <string>push</string>     </void>
    <void property="actionArgument"> <string>tv</string>       </void>
  </object>
  <object class="org.jdesktop.lg3d.apps.smrt.menu.IconItem">
    <void property="iconFilename">   <string>music.png</string> </void>
    <void property="label">
        <string>Listen to Music</string>
    </void>
    <void property="action">         <string>push</string>      </void>
    <void property="actionArgument"> <string>music</string>     </void>
  </object>
</java>
\end{code}
\end{slide}


\begin{slide}{Applications Configuration}
\begin{itemize}
        \item Want to leverage any existing player applications
	\item Different players for different media types -- no existing player can do everything
	\item \textit{applications.conf} associates regular expressions with ApplicationFactory objects
	\item Regular expression is matched against the argument given to \textbf{launch} actions
\end{itemize}
\end{slide}

\begin{slide}[method=direct]{applications.conf}
\begin{code}
<java version="1.5.0" class="java.beans.XMLDecoder">
  <object class="org.jdesktop.lg3d.apps.smrt.application.XineFactory">
    <void property="regexs">
      <array class="java.lang.String">
        <string>.*\.avi</string>
        <string>.*\.mpeg</string>
        <string>dvd\://.*</string>
      </array>
    </void>
  </object>
  <object class="org.jdesktop.lg3d.apps.smrt.application.MPlayerFactory">
    <void property="regexs">
      <array class="java.lang.String">
        <string>tv\://p{Digit}*</string>
      </array>
    </void>
  </object>
</java>
\end{code}
\end{slide}

\presenter{Daniel Seikaly}

% Summary of Major Points
\section[slide=false]{Summary}
\begin{slide}[toc=,bm=]{Summary}
\begin{itemize}
\item The Problem: Media Center in 3D
\item The Solution: smrt \\
\begin{itemize}
	\item Requirements
	\item User Interface and Demonstration
	\item Architecture
\end{itemize}
\end{itemize}
\end{slide}

\presenter{}
\section[slide=false]{Questions}
\begin{slide}[toc=,bm=]{Thanks!}
Thank you!
\end{slide}

\begin{slide}[toc=,bm=]{Questions}
\begin{figure}[htb]
	\includegraphics[width=3in]{../lib/figures/fishstick}
\end{figure}
\end{slide}

\end{document}
