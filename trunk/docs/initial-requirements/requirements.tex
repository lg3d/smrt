\documentclass[letterpaper, titlepage, 11pt]{article}
\usepackage{fullpage}
\usepackage{graphicx}
\usepackage{hyperref}
\usepackage{url}
\usepackage{titling}

% This is here so we can have a fancier title page than LaTeX gives us by default
\newcommand{\department}[1]{%
  \gdef\dept{#1}}
\newcommand{\dept}{}
\renewcommand{\maketitlehookd}{%
\par\noindent \dept }

\title{
	smrt: A 3D Media Center User Interface
	\\
	Initial Requirements
}
\author{
	Cory Maccarrone  \\ {\small \href{mailto:Cory.Maccarrone@colorado.edu}{Cory.Maccarrone@colorado.edu}}
\and
	Daniel Seikaly   \\ {\small \href{mailto:Daniel.Seikaly@colorado.edu}{Daniel.Seikaly@colorado.edu}}
\and
	Evan Sheehan     \\ {\small \href{mailto:Wallace.Sheehan@gmail.com}{Wallace.Sheehan@gmail.com}}
\and
	David Trowbridge \\ {\small \href{mailto:trowbrds@gmail.com}{trowbrds@gmail.com}}
}
\date{September 21, 2005}
\department{
\begin{center}
	CSCI 4308-4318. Software Engineering Project 1 \& 2 \\
	Department of Computer Science \\
	University of Colorado at Boulder \\
	2005-2006 \\
	\vspace{1.5em}
	Sun Microsystems \\
	Santa Clara, CA \\
	\vspace{1em}
	Paul Byrne \\
	{\small \href{mailto:Paul.Byrne@Sun.COM}{Paul.Byrne@Sun.COM}} \\
	\vspace{1em}
	Hideya Kawahara \\
	{\small \href{mailto:Hideya.Kawahara@Sun.COM}{Hideya.Kawahara@Sun.COM}}
\end{center}
}

\begin{document}
\maketitle

\raggedbottom

\pagenumbering{roman}

\hspace{1em}
\pagebreak

\tableofcontents
\listoffigures
\pagebreak

\hspace{1em}
\pagebreak


\pagenumbering{arabic}

\documentclass[letterpaper, notitlepage, 11pt]{article}
\usepackage[body={6in, 8in}, left=1in, right=1in, top=1in, bottom=1in]{geometry}
\usepackage{fancyhdr}

\pagestyle{empty}

\begin{document}
\documentclass[letterpaper, notitlepage, 11pt]{article}
\usepackage[body={6in, 8in}, left=1in, right=1in, top=1in, bottom=1in]{geometry}
\usepackage{fancyhdr}

\pagestyle{empty}

\begin{document}
\documentclass[letterpaper, notitlepage, 11pt]{article}
\usepackage[body={6in, 8in}, left=1in, right=1in, top=1in, bottom=1in]{geometry}
\usepackage{fancyhdr}

\pagestyle{empty}

\begin{document}
\input{../lib/project-proposal}
\end{document}

\end{document}

\end{document}

\section{Introduction}
As a company, one of Sun Microsystems' objectives is to innovate the world of
computing. To this end, Sun created Project Looking Glass to explore the field
of 3D user interfaces and determine what improvements in user interaction can be
made by taking advantage of the third dimension. Through Project Looking Glass,
Sun hopes to begin redefining how people think of user interfaces and create
useful design concepts for a 3D computing environment. At the moment, Looking
Glass consists of a framework for developing 3D applications and a desktop
environment to run them alongside existing 2D applications.

The goal of this project, code named \textit{smrt}, is to create a user
interface for a home media center along the lines of TiVo, but using 3D user
interface elements within the Looking Glass environment. The name \textit{smrt}
-- pronounced ``smeert'' -- is the Czech word for ``death,'' and was primarily
chosen because it is fun to say and spell.

Figure \ref{figure:concept} presents a conceptual diagram of the overall
system.  This diagram shows how \textit{smrt} interacts with its software and
hardware environment. At the most basic level, \textit{smrt} allows a user to
browse through and play media, as well as watch or record a TV show.  To control
the system, a simple input device such as keyboard or remote control is used.
Note that this project is focused on the user interface; actual functionality
may not exist.

\begin{figure}[htb]
\centering
\includegraphics[width=4in]{figures/conceptual_overview}
\caption{Conceptual overview of the \textit{smrt} project\label{figure:concept}}
\end{figure}

This document presents a preliminary list of requirements which will be used to develop
a high-level design for \textit{smrt}.  Once the design is created, it will be used in the
development of the prototype.  These requirements do not necessarily represent requirements
traceable through project completion.  A more complete document will be assembled at a later
time.

\section{Requirements}
Requirements for \textit{smrt} encompass development environment, hardware,
functional, documentation, and release specifications. Some of these
specifications were given directly by the sponsors; the rest were derived from
other requirements.

\pagebreak

\subsection{Supporting Environment}
The Supporting Environment includes both the hardware and software environments
within which \textit{smrt} must work. These requirements pertain to both the
environment in which \textit{smrt} is developed and the environment in which it
runs.

\subsubsection{Software}
\begin{itemize}
\item Runs within the latest stable build of the Project Looking Glass environment (currently 0.7.0).
\item Software is written using Java, running under the Sun JDK version 5.0 or higher.
\end{itemize}

\subsubsection{Hardware}
\begin{itemize}
\item Must be demonstrable on hardware supporting TV/HDTV video output.
\item Support for input devices. Minimally, it must work with a remote control
      that has 4 directional buttons, an enter button and an escape button (or
      unused button which can be used as escape).
\item Extensible to esoteric and future input devices, such as remote controls
      with analog joysticks.  The only input devices which must be inherently
      supported are a basic remote control or a keyboard.
\item Utilize permanent storage, such as a hard drive or FLASH memory for video
      storage.
\end{itemize}

\subsection{Functional Requirements}
Functional Requirements specify all of the functionality that \textit{smrt} is required
to provide.  This includes display capabilities, user interaction, scalability, and
user interface breadth.

\subsubsection{Display}
\begin{itemize}
\item The software must be able to output at multiple video resolutions. At the low
      end of the scale, it must support standard NTSC televisions.  At the high end,
      it must support at least 720p \footnote{An HDTV signal type which displays using
      progressive scanning at 720 lines of resolution.  Actual pixel resolution varies
      by television, but is typically 1280x720.}.
\end{itemize}

\subsubsection{User Interaction and Scalability}
\begin{itemize}
\item The system must make use of 3D user interface elements in order to accelerate the
      interaction with the system.
\item The system must scale to large media collections.  For the purposes of traceability,
      ``large'' will be defined as a music collection with at least 300 albums and over
      two thousand individual songs or over 200 movies.
\item The system must minimize user interaction to browse large media
	collections.  It must require noticeably fewer keyboard or
	remote-control actions than the normal 2D user interface of the chosen
	media center back-end (or, if no preexisting back-end is used, fewer
	actions than other media centers such as freevo or MythTV).
\end{itemize}

\subsubsection{Minimum User Interface Breadth}
These are the user interface requirements.  Note that this only reflects the user interface;
the actual functionality may or may not work -- e.g., the system will allow you to schedule a
recording, but the program may not actually be recorded, depending on whether time to implement
such a back-end is available.
\begin{itemize}
\item The system must allow the user to browse and select media to play.
\item The system must allow the user to select a TV channel to watch.
\item The system must allow the user to schedule a TV program to record.
\end{itemize}

\subsubsection{Other User Interface Possibilities}
These are additional features we could potentially add to the UI if time allows, but are
not part of the minimum requirements for the software.
\begin{itemize}
\item Management of media -- allowing the user to delete, move, rename, fetch meta-data, etc.
\item Browsing through media via ``live search'' - search by artist, actor, year, etc.
\item Resolving conflicts between scheduled recordings.
\end{itemize}

\subsection{Documentation and Release}
\begin{itemize}
\item Written documentation will be supplied in PDF format.
\item All documentation and source code will be released through a java.net sub-project.
\end{itemize}

\begin{flushleft}
The following documentation artifacts will be produced:
\end{flushleft}

\begin{itemize}
\item Requirements document
\item System architecture document
\item UI design decisions document
\item Installation and use instructions
\item Developer documentation via Javadoc
\end{itemize}

\section{Summary}
As a sub-project of Project Looking Glass, \textit{smrt}'s goal is to explore the
possibilities of a 3D user interface when applied to a home media center. The project
is less focused on creating a working media center then it is on innovating the realm
of 3D user interfaces and set top boxes. It is the hope of the project sponsors that
some useful interaction and design concepts will come from this project.

\end{document}
