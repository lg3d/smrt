\documentclass[letterpaper, titlepage, 11pt]{article}
\usepackage{fullpage}
\usepackage{graphicx}
\usepackage{hyperref}
\usepackage{url}
\usepackage{titling}

% This is here so we can have a fancier title page than LaTeX gives us by default
\newcommand{\department}[1]{%
  \gdef\dept{#1}}
\newcommand{\dept}{}
\renewcommand{\maketitlehookd}{%
\par\noindent \dept }

\title{
	smrt: A 3D Media Center User Interface
	\\
	Requirements
}
\author{
	Cory Maccarrone  \\ {\small \href{mailto:Cory.Maccarrone@colorado.edu}{Cory.Maccarrone@colorado.edu}}
	\and
	Daniel Seikaly   \\ {\small \href{mailto:Daniel.Seikaly@colorado.edu}{Daniel.Seikaly@colorado.edu}}
	\and
	Evan Sheehan     \\ {\small \href{mailto:Wallace.Sheehan@gmail.com}{Wallace.Sheehan@gmail.com}}
	\and
	David Trowbridge \\ {\small \href{mailto:trowbrds@gmail.com}{trowbrds@gmail.com}}
}
\date{November 2, 2005}
\department{
\begin{center}
	CSCI 4308-4318. Software Engineering Project 1 \& 2 \\
	Department of Computer Science \\
	University of Colorado at Boulder \\
	2005-2006 \\
	\vspace{1.5em}
	Sun Microsystems \\
	Santa Clara, CA \\
	\vspace{1em}
	Paul Byrne \\
	{\small \href{mailto:Paul.Byrne@Sun.COM}{Paul.Byrne@Sun.COM}} \\
	\vspace{1em}
	Hideya Kawahara \\
	{\small \href{mailto:Hideya.Kawahara@Sun.COM}{Hideya.Kawahara@Sun.COM}}
\end{center}
}

\begin{document}
\maketitle

\raggedbottom

\pagenumbering{roman}

\hspace{1em}
\pagebreak

\tableofcontents
\listoffigures
\pagebreak

\hspace{1em}
\pagebreak

\pagenumbering{arabic}

\section{Project Proposal}
Project Looking Glass is an open source development project based on and evolved
from Sun Microsystems' advanced technology project. It supports running unmodified
existing applications in a 3D space, as well as APIs for 3D window manager and
application development. Project Looking Glass has great potential for allowing
users to look at data in new ways from which they will be able to extract more
information.

The goal of this project is to explore the use of a 3D user interface for home media
centers for use with a TV (and possibly HDTV) display -- think 3D user interface for
TiVo. The main objective is to explore the UI possibilities, so this could be a
mockup or the team may want to integrate with an existing system such as MythTV.

\begin{flushleft}
Interesting areas to explore would be:
\begin{itemize}
\item the selection of programs to record
\item managing conflict of scheduled recordings
\item selection of programs to play
\item how the third dimension could be used to blend other features with the playing of a TV program
\end{itemize}
\end{flushleft}

Although the target hardware for such a system would be a set top box, we'd suggest
for this research the students use standard PCs, but constrain the resolution to TV
and HDTV quality.

\section{Introduction}
As a company, one of Sun Microsystems' objectives is to innovate the world of
computing. To this end, Sun created Project Looking Glass to explore the field
of 3D user interfaces and determine what improvements in user interaction can be
made by taking advantage of the third dimension. Through Project Looking Glass,
Sun hopes to begin redefining how people think of user interfaces and create
useful design concepts for a 3D computing environment. At the moment, Looking
Glass consists of a framework for developing 3D applications and a desktop
environment to run them alongside existing 2D applications.

The goal of this project, code named \textit{smrt}, is to create a user
interface for a home media center along the lines of TiVo, but using 3D user
interface elements within the Looking Glass environment. The name \textit{smrt}
-- pronounced ``smeert'' -- is the Czech word for ``death,'' and was primarily
chosen because it is fun to say and spell.

Figure \ref{figure:concept} presents a conceptual diagram of the overall
system.  This diagram shows how \textit{smrt} interacts with its software and
hardware environment. At the most basic level, \textit{smrt} allows a user to
browse through and play media, as well as watch or record a TV show.  To control
the system, a simple input device such as keyboard or remote control is used.
Note that this project is focused on the user interface; actual functionality
may not exist.

\begin{figure}[htb]
\centering
\includegraphics[width=4in]{figures/conceptual_overview}
\caption{Conceptual overview of the \textit{smrt} project\label{figure:concept}}
\end{figure}

Details regarding the requirements of \textit{smrt} are presented in the next
section, including some alternative requirements for the project. Following the
project's requirements is a section outlining some features that are not
strictly required, but should be mentioned as possibilities for future
development. A glossary of terms and a list of related documents can be found at
the end of this document.

\section{Requirements}
\label{sec:requirements}
This section describes the requirements for the \textit{smrt} software package.
Because these requirements are fairly extensive and cover a variety of different
aspects of the system, the discussion has been divided into several logical
sections.  These sections discuss the development and runtime environments,
hardware and functional requirements, as well as a specification for
documentation and release.

\pagebreak

\subsection{Supporting Environment}
The Supporting Environment includes both the hardware and software environments
in which \textit{smrt} must function. These specifications include requirements
for the development environment as well as the runtime environment.

\subsubsection{Software}
\textit{smrt} is a sub-project of Sun's Project Looking Glass, which means that
\textit{smrt} needs to run with the latest stable build of the Looking Glass
environment -- currently this is version 0.7.1. Compatibility with Looking Glass
also requires that \textit{smrt} be written in Java and run under the Sun JDK
(Java Development Kit) version 5.0 or higher.

% To run, \textit{smrt} requires OpenGL for 3D acceleration. The X Window System
% is required for video playback because \textit{smrt} makes use of the X11
% integration provided by Project Looking Glass.

\subsubsection{Hardware}
Because \textit{smrt} is a media center application that is intended for use
with televisions, it must be demonstrable on hardware supporting TV/HDTV video
output. In order to function, \textit{smrt} must accept user input. As a media
center application this would realistically be a remote control; however, for
the purpose of this project all that is required is keyboard input. The type of
input from the keyboard should be analogous to the type of input one would
expect from a remote -- i.e. four directional buttons, an enter button, and an
escape button.

\subsection{Functional Requirements}
The Functional Requirements encompass all of the functionality that
\textit{smrt} is required to provide. It specifies display capabilities, user
interface breadth, user interaction, and scalability.

\subsubsection{Display}
The graphical user interface must scale to multiple video resolutions so that it is
viewable on different televisions or monitors. At the low end of the scale, it
must support standard NTSC televisions. At the high end, it must support at
least 720p\footnote{An HDTV signal which display using progressive scanning at
720 lines of resolution. Actual pixel resolution varies by television, but is
typically 1280x720.}.

\subsubsection{Minimum User Interface Breadth}
As a media center user interface, \textit{smrt} needs to provide mechanisms for
the user to access all the media that they have available. It should allow
the user to browse and select locally stored media for playback. In addition to
locally stored files, \textit{smrt} should provide access to removable media
such as DVDs and audio CDs. The user should also be able to select a TV channel
to watch, and schedule TV programs for recording.

In order to browse media, \textit{smrt} needs an interface for accessing local
files on the machine. When the user selects a media file whose type is supported
by \textit{smrt}, \textit{smrt} should load that file in the appropriate media
player. When playback has completed \textit{smrt} should return to the screen
the user saw before selecting the media file. To reduce the amount of
information on the screen, \textit{smrt} will need to be able to filter the
displayed files by type.

\textit{smrt} should be able to obtain up to date program information for the TV
channels that the user has access to. It should display this information and
allow the user to select programs or channels to watch. In addition to viewing
programs and channels, the user should be able to mark certain programs for
recording. Scheduling a recording can be handled in one of two ways: the user
can select a program by name for recording, or specify a channel, start time,
and duration for the recording.

\subsubsection{User Interaction and Scalability}
One of the aims of Project Looking Glass is to experiment and innovate in
the realm of user interfaces.  To this end, \textit{smrt} must use 3D user
interface elements to accelerate the user's interaction with the system.

Because \textit{smrt} can be used to access media stored locally, it needs to
scale gracefully to large media collections. For the purposes of traceability,
``large'' will be defined as a music collection with at least 300 albums and
over 2,000 individual songs, or over 200 movies. For browsing collections this size
\textit{smrt} needs to minimize user interaction. It should take noticeably
fewer keyboard or remote-control actions than a standard 2D user interface for
the task.

\subsection {Deliverables}
There are several deliverables which will be presented to the sponsors at the
end of this project.  These include several documents, relating to both
development and usage, as well as the source code to the software.

\subsubsection{Documentation}
Several documents will be produced, including:
\begin{itemize}
\item Requirements specification
\item Design specification
\item Installation and use instructions: this document will explain how to
      install, configure and use \textit{smrt}.  It will enumerate the
      software dependencies and explain their installation process.  In
      addition, it will explain the file format for creating menus.
\item UI design decisions document: this document will be an explanation of
      the user interfaces in \textit{smrt} and the reasons that certain choices
      were made.  It will explain how the various 3D user interface elements
      work, both in terms of user interaction and code architecture.
\item Developer documentation via Javadoc
\end{itemize}

\subsubsection{Release}
All documentation and source code will be released through a java.net
sub-project.  Written documentation will be released either as PDF or
HTML as appropriate.

\subsection{Alternative Requirements}
Initially, we were going to require that \textit{smrt} be written in Python. It
was our intent to use Freevo (an existing open source home theater application)
as a backend to provide \textit{smrt}'s functionality. Because Freevo is written
in Python, \textit{smrt} would also need to be written in Python. However, since
\textit{smrt} needs to use Looking Glass, it cannot be written in Python.

While discussing input devices, we considered a variety of different types of
remote controls. We considered remotes that have arrow keys, joy sticks, and
even accelerometers. Ultimately our sponsor told that we are not required to
support remote controls, a keyboard is all that they require.

\section{Future Enhancements}
\label{sec:enhancements}
The above requirements outline a minimum level of functionality and performance
for \textit{smrt}; they are not, however, exhaustive. Below are some features
that may not be implemented in the course of this project, but should be taken
into consideration regardless.

\subsection{Media Management}
In addition to allowing the user to browse the locally stored media,
\textit{smrt} might also provide mechanisms for managing the media, i.e. allow
the user to delete, move, rename, fetch meta-data, etc. All of these tasks can
be accomplished using other tools readily available to the user, but would be
more convenient if accessible from \textit{smrt}.

\subsection{Searching}
The ability to search media with a ``live search'' would improve \textit{smrt}'s
browsing capabilities. It should allow the user to search by artist, actor,
year, etc. To achieve this, \textit{smrt} will need a means of searching through
meta-data where available and fallback to file names when it isn't.

\subsection{Resolving Scheduling Conflicts}
When a user schedules a TV program for recording, \textit{smrt} should be able
to detect if this recording conflicts with a previously scheduled recording. Due
to the nature of the recording functionality, only one program may be recorded
at a time. So if there is a conflict, the user needs to be informed and asked to
resolve it.

\subsection{Remote Controls}
Normally a user will be using a remote control, not a keyboard, to interact with
their media center. \textit{smrt} should support remote controls as input
devices. At a minimum, the remote will need four directional buttons, an enter
button, and an escape button. In addition to this basic remote, \textit{smrt}
will need to be expandable to more elaborate types of remotes, such as remotes
with joysticks or even accelerometers.

\section{Summary}
As a sub-project of Project Looking Glass, \textit{smrt}'s goal is to explore
the possibilities of a 3D user interface when applied to a home media center.
The project is less focused on creating a working media center then it is on
innovating the realm of 3D user interfaces and set top boxes. It is the hope of
the project sponsors that some useful interaction and design concepts will come
from this project.

In this paper we have enumerated the requirements for this project. Section
\ref{sec:requirements} explains all of the developmental and functional
requirements for \textit{smrt}, as well as identifying the documents that will
accompany this project. At the end of section \ref{sec:requirements}, we briefly
discuss alternate requirements that we considered early on in the life of the
project and subsequently rejected. Section \ref{sec:requirements} explains only
the minimum of what this project requires, as a result there are features not
identified here that should still be considered for this project. These extra
features are discussed in section \ref{sec:enhancements}. Once the minimum
requirements have been met, these features will continue to guide
\textit{smrt}'s development. As a whole, this document should provide enough
information to develop a design for \textit{smrt} and evaluate the
implementation of that design.

\pagebreak
\appendix
\section{Glossary}
A list of terms and acronyms used in this document, along with their
definitions, is provided below.

\begin{tabular}{p{.3\linewidth}p{.6\linewidth}}
	\textbf{API}			& Application Programming Interface	\\
	\textbf{CD}			& Compact Disc				\\
	\textbf{DVD}			& Three letters referring to a type of
					  optical disc storage. (In 1999 the DVD
					  Forum decreed that DVD is not an
					  acronym for anything.)		\\
	\textbf{HDTV}			& High Definition TeleVision		\\
	\textbf{HTML}			& HyperText Markup Language		\\
	\textbf{Java}			& The Java programming language		\\
	\textbf{JDK}			& The Java Development Kit		\\
	\textbf{MythTV}			& An open source application for
					  running a home theater system on a
					  computer.				\\
	\textbf{NTSC}			& National TeleVision System Committee.
					  The group responsible for setting
					  TeleVision and video standards.	\\
	\textbf{PDF}			& Portable Document Format		\\
	\textbf{Project Looking Glass}	& Sun Microsystems' experimental 3D
					  desktop environment framework		\\
	\textbf{smrt}			& The Czech word for ``death''. Also
					  the name of this project: a rocking
					  3D home theater interface.		\\
	\textbf{TV}			& TeleVision				\\
	\textbf{TiVo}			& A commercial digital video recorder.	\\
	\textbf{XML}			& Extensible Markup Language		\\
\end{tabular}

\section{Related Documents}
\begin{list}{}{
\setlength{\parsep}{1ex}
\setlength{\leftmargin}{0.5in}
\setlength{\itemindent}{-0.5in}
}

\item[] \textbf{[Gosling 05]}

	Gosling, James, Bill Joy, Guy Steele, and Gilad Bracha. \textit{The Java
	Language Reference: Third Edition}. Addison-Wesley, Santa Clara, California, 2005.

	The third edition of the Java language specification.

\item[] \textbf{[Bray 04]}

	Bray, Tim, John Cowan, Eve Maler, et al. \textit{Extensible Markup
	Language (XML) 1.1}. W3C, 2004.

	The World Wide Web Consortium's most recent recommendation regarding the
	XML specification.

\item[] \textbf{[Project Looking Glass]}

	http://lg3d-core.dev.java.net/

	The open source web site for Sun's Project Looking Glass.
\end{list}

\end{document}
