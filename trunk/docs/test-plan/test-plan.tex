\documentclass[letterpaper, titlepage, 11pt]{article}
\usepackage[usenames]{color}
\usepackage{fullpage}
\usepackage{graphicx}
\usepackage{hyperref}
\usepackage{url}
\usepackage{titling}
\usepackage{ifthen}

% This is here so we can have a fancier title page than LaTeX gives us by default
\newcommand{\department}[1]{%
  \gdef\dept{#1}}
\newcommand{\dept}{}
\renewcommand{\maketitlehookd}{%
\par\noindent \dept }

\newcommand{\testitem}[8]{%
  \ifthenelse{\equal{#1}{}}{}{\subsubsection{#1}}
  \begin{tabular}{lp{4in}}
  \textbf{Purpose}         & {#2} \\
  \textbf{Procedure}       & {#3} \\
  \textbf{Expected Result} & {#4} \\
  \textbf{Comments}        & {#5} \\
  \textbf{Date}            & {#6} \\
  \textbf{Tester}          & {#7} \\
  \textbf{Outcome}         &
    \ifthenelse{\equal{#8}{pass}}{\textcolor{green}{Pass}}{%
      \ifthenelse{\equal{#8}{fail}}{\textcolor{red}{Fail}}{#8}
    } \\
  \end{tabular}
}

\title{
	smrt: A 3D Media Center User Interface
	\\
	Test Plan
}
\author{
	Cory Maccarrone  \\ {\small \href{mailto:Cory.Maccarrone@colorado.edu}{Cory.Maccarrone@colorado.edu}}
	\and
	Daniel Seikaly   \\ {\small \href{mailto:Daniel.Seikaly@colorado.edu}{Daniel.Seikaly@colorado.edu}}
	\and
	Evan Sheehan     \\ {\small \href{mailto:Wallace.Sheehan@gmail.com}{Wallace.Sheehan@gmail.com}}
	\and
	David Trowbridge \\ {\small \href{mailto:trowbrds@gmail.com}{trowbrds@gmail.com}}
}
\department{
\begin{center}
	CSCI 4308-4318. Software Engineering Project 1 \& 2 \\
	Department of Computer Science \\
	University of Colorado at Boulder \\
	2005-2006 \\
	\vspace{1.5em}
	Sun Microsystems \\
	Santa Clara, CA \\
	\vspace{1em}
	Paul Byrne \\
	{\small \href{mailto:Paul.Byrne@Sun.COM}{Paul.Byrne@Sun.COM}} \\
	\vspace{1em}
	Hideya Kawahara \\
	{\small \href{mailto:Hideya.Kawahara@Sun.COM}{Hideya.Kawahara@Sun.COM}}
\end{center}
}

\begin{document}
\maketitle

\raggedbottom

\pagenumbering{roman}

\hspace{1em}
\pagebreak

\tableofcontents
\listoffigures
\pagebreak

\hspace{1em}
\pagebreak

\pagenumbering{arabic}

\documentclass[letterpaper, notitlepage, 11pt]{article}
\usepackage[body={6in, 8in}, left=1in, right=1in, top=1in, bottom=1in]{geometry}
\usepackage{fancyhdr}

\pagestyle{empty}

\begin{document}
\documentclass[letterpaper, notitlepage, 11pt]{article}
\usepackage[body={6in, 8in}, left=1in, right=1in, top=1in, bottom=1in]{geometry}
\usepackage{fancyhdr}

\pagestyle{empty}

\begin{document}
\documentclass[letterpaper, notitlepage, 11pt]{article}
\usepackage[body={6in, 8in}, left=1in, right=1in, top=1in, bottom=1in]{geometry}
\usepackage{fancyhdr}

\pagestyle{empty}

\begin{document}
\input{../lib/project-proposal}
\end{document}

\end{document}

\end{document}

\section{Introduction}
As a company, one of Sun Microsystems' objectives is to innovate the world of
computing. To this end, Sun created Project Looking Glass to explore the field
of 3D user interfaces and determine what improvements in user interaction can be
made by taking advantage of the third dimension. Through Project Looking Glass,
Sun hopes to begin redefining how people think of user interfaces and create
useful design concepts for a 3D computing environment. At the moment, Looking
Glass consists of a framework for developing 3D applications and a desktop
environment to run them alongside existing 2D applications.

The goal of this project, code named \textit{smrt}, is to create a user
interface for a home media center along the lines of TiVo, but using 3D user
interface elements within the Looking Glass environment. The name \textit{smrt}
-- pronounced ``smeert'' -- is the Czech word for ``death,'' and was primarily
chosen because it is fun to say and spell.

Figure \ref{figure:concept} presents a conceptual diagram of the overall
system.  This diagram shows how \textit{smrt} interacts with its software and
hardware environment. At the most basic level, \textit{smrt} allows a user to
browse through and play media, as well as watch or record a TV show.  To control
the system, a simple input device such as keyboard or remote control is used.
Note that this project is focused on the user interface; actual functionality
may not exist.

\begin{figure}[htb]
\centering
\includegraphics[width=4in]{figures/conceptual_overview}
\caption{Conceptual overview of the \textit{smrt} project\label{figure:concept}}
\end{figure}

This document defines the test plan for \textit{smrt}. These tests will
demonstrate that \textit{smrt} meets the requirements defined in the
\textit{Requirements Specification}. The hardware and software environments are
defined in section \ref{sec:environment}, followed by a definition of all the
tests. Each test definition provides a purpose, procedure, and expected result.
The purpose explains the reason for running this particular test. The procedure
defines the steps for running the test. And the expected result defines the
conditions for passing the test. In addition to the fields that define the
test, each test has the following fields: comments, date, tester, and outcome.
These fields provide information about the results of running the test.

\section{Testing Environment}
\label{sec:environment}
Hardware and software requirements for running the tests are outlined below.

\subsection{Software}
\textit{smrt}'s test environment requires the following software:
\begin{itemize}
\item Linux
\item X11 window system with Direct Rendering Infrastructure (DRI)
\item Java 5.0 or higher
\item Looking Glass 3D 0.7.1 or higher
\end{itemize}

\subsection{Hardware}
\textit{smrt}'s hardware requirements are fairly minimal:
\begin{itemize}
\item Input device -- i.e. keyboard
\item Display -- i.e. a monitor or TV
\end{itemize}

\section{Tests}
Tests are organized into groups that target specific areas of the program.

Each test in the test plan has several components:
\newline
\testitem{}
  {the reason for the test.}
  {the steps to follow to conduct the test.}
  {the results necessary to pass the test.}
  {any comments the tester might have.}
  {date the test was conducted.}
  {name of the person conducting the test.}
  {outcome of the test (\textcolor{green}{Pass} or \textcolor{red}{Fail}).}

\subsection{Display Resolution Tests}
The display resolution tests verify that \textit{smrt} can adapt to multiple
displays so that it is viewable on different televisions and monitors.
Completion of these tests is considered to verify the display resolution
requirement set in section 3.2.1 of \textit{smrt}: Requirements.

\testitem{Ring Menu Active Resize}
  {
    This test determines if the ring menu behaves correctly with different
    screen resolutions.
  }
  {
    Start \textit{smrt}.  Browse to a ring menu if the main menu is not a
    ring.  Resize the window.
  }
  {
    When first displayed, the ring menu should fit on the screen.  As the
    window is resized, the ring menu should scale to fit the new resolution.
  }
  {}{May 5 2006}{David Trowbridge}{pass}

\testitem{Ring Menu Passive Resize}
  {
    This test determines if the ring menu behaves correctly when a resolution
    change occurs in a child menu from the ring menu.
  }
  {
    Start \textit{smrt}.  Browse to a ring menu if the main menu is not a
    ring.  Select an item in the menu to move to another menu.  Resize
    the window.  Escape back to the ring menu.
  }
  {
    When resized, the ring menu should fit the new dimensions correctly. Also,
    the Heads Up Display (HUD), the menu bars at the top and bottom of the 
    display, will appear at the proper edges of the screen, and all items will be
    sized according to new window dimensions.
  }
  {}{May 5 2006}{David Trowbridge}{pass}

\testitem{Arc Menu Active Resize}
  {
    This test determines if the arc menu behaves correctly with different
    screen resolutions.
  }
  {Start \textit{smrt}.  Browse to an arc menu.  Resize the window.}
  {
    When first displayed, the arc menu should fit on the screen.  As the
    window is resized, the arc menu should scale to fit the new resolution.
  }
  {}{May 5 2006}{David Trowbridge}{pass}

\testitem{Arc Menu Passive Resize}
  {
      This test determines if the arc menu behaves correctly when a resolution
      change occurs in a child menu from the arc menu.
  }
  {
    Start \textit{smrt}.  Browse to an arc menu.  Select an item in the
    menu to move to another menu.  Resize the window.  Escape back to the
    arc menu.
  }
  {
    When resized, the arc menu should fit the new dimensions correctly,
    with the HUD at the proper edges of the screen and all items sized
    according to new window dimensions.
  }
  {Requires changing the default config}{May 5 2006}{David Trowbridge}{pass}

\testitem{Cityscape Menu Active Resize}
  {
    This test determines if the cityscape behaves correctly with different
    screen resolutions.
  }
  {Start \textit{smrt}.  Browse to a cityscape.  Resize the window.}
  {
    When first displayed, the cityscape should show an easily-readable
    number of items on the screen (dependent on cityscape contents).  As
    the window is resized, it should scale to fit the new resolution while
    displaying the same picture.
  }
  {Requires changing the default config}{May 5 2006}{David Trowbridge}{pass}

\testitem{Cityscape Menu Passive Resize}
  {
    This test determines if the cityscape menu behaves correctly when a
    resolution change occurs in a child menu from the cityscape menu.
  }
  {
    Start \textit{smrt}.  Browse to a cityscape.  Select an item in the
    menu to move to another menu.  Resize the window.  Escape back to the
    cityscape menu.
  }
  {
    When resized, the cityscape menu should fit the new dimensions correctly,
    with the HUD at the proper edges of the screen and all items sized
    according to new window dimensions.
  }
  {}{May 5 2006}{David Trowbridge}{pass}

\testitem{Background Resize}
  {
    This test determines if the background image behaves correctly with
    different screen resolutions.
  }
  {Start \textit{smrt}.  Resize the window.}
  {
    When first displayed, the background image should fit the screen exactly.
    As the window is resized, it should scale to fit the new resolution.
  }
  {}{May 5 2006}{David Trowbridge}{pass}

\testitem{Application Resize}
  {
    This test determines if application windows behave correctly with
    different screen resolutions.
  }
  {Start \textit{smrt}.  Browse to a file and launch it.  Resize the window.}
  {
    When first displayed, the application window should fill the screen.
    As the window is resized, the application should scale to fit the
    new resolution.
  }
  {}{May 5 2006}{David Trowbridge}{pass}

\subsection{Basic Interaction Tests}
Basic interaction tests verify \textit{smrt}'s responses to user input. These
tests cover the requirements in section 3.2.2 of \textit{smrt}: Requirements.

\testitem{Ring Menu Interaction}
  {
    This test determines if the ring menu is functioning properly.
  }
  {
    Start \textit{smrt}.  Browse to a ring menu which is not the main menu.
    Press left and right to spin the ring.  Press enter to choose an item.
    Press escape to return to the ring menu, then escape again to destroy it.
  }
  {
    The left key should spin the ring menu so that the element to the left
    of the currently-selected item rotates to become selected.  Similarly, the
    right key should spin the menu to select the item to the right.  The enter
    key should select the current item, and the escape key should go back to
    the previous menu.
  }
  {
    Last test (popping) requires changing the default config
  }
  {May 5 2006}{David Trowbridge}{pass}

\testitem{Arc Menu Interaction}
  {
    This test determines if the arc menu is functioning properly.
  }
  {
    Start \textit{smrt}.  Browse to an arc menu.  Press up and down to move
    the arc.  Press right and left to page through the arc.  Press enter to
    choose an item.  Press escape to return to the arc menu, then escape again
    to destroy it.
  }
  {
    The up key should move the arc to select the element above the currently
    selected item.  Similarly, the down key should select the item below.  The
    right key should "page down" through the arc, and left should "page up."
    The enter key should select the current item, and the escape key should go
    back to the previous menu.
  }
  {}{May 5 2006}{David Trowbridge}{pass}

\testitem{Cityscape Menu Interaction}
  {
    This test determines if the cityscape menu is functioning properly.
  }
  {
    Start \textit{smrt} and browse to a cityscape menu. Press up, down, left,
    and right to navigate the cityscape. Press enter while a directory is
    selected. Press escape to return to the previous cityscape, then press
    escape again to exit the cityscape.
  }
  {
    The direction keys should select the next item in that direction, if there
    is one -- i.e. up should select the item immediately above the current item,
    right should select the item to the right, etc. When a directory is selected
    the view should zoom in on that building and display its contents as a
    cityscape menu. The escape key should return to the previous menu.
  }
  {}{May 5 2006}{David Trowbridge}{pass}

\testitem{Application Launching}
  {
    This test verifies that applications are correctly launched when certain
    files are selected from a cityscape menu.
  }
  {
    Navigate to a supported file type (e.g. \texttt{.avi}, or \texttt{.mp3}) in
    a cityscape menu. Press enter to activate the file.
  }
  {
    Xine should be started full-screen. It should immediately begin playing the
    file.
  }
  {}{May 5 2006}{David Trowbridge}{pass}

\testitem{Application Interaction}
  {
    This test verifies that the user can interact with external applications
    when they are launched inside of \textit{smrt}.
  }
  {
    Launch a video file in Xine. Press the left and right keys. Press the up and
    down keys. Press the space bar. Press escape.
  }
  {
    The right key should fast-forward, the left key should rewind the video. Up
    should increase playback speed, down should decrease playback speed. The
    space bar should pause playback. Escape should quit Xine and return to the
    previous menu.
  }
  {}{May 5 2006}{David Trowbridge}{pass}

\testitem{Television Interaction}
  {
    This test determines if the required functionality for watching TV is
    working correctly.
  }
  {
    Navigate to a \texttt{TvArcMenu}. Press the up and down keys a few times.
    Press enter on a channel to select it. Press escape to return to the menu.
  }
  {
    Pressing up and down should move the menu up and down, changing the
    selection. When the selection changes, the program information and the
    preview should change to the currently selected channel. Pressing enter
    should launch an application to display the TV channel full-screen. Pressing
    escape should quit the application and return to the menu with the
    previously selected channel centered in the menu.
  }
  {}{May 5 2006}{David Trowbridge}{pass}

\subsection{Scalability Tests}
Scalability tests verify \textit{smrt}'s ability to handle large and small media
collections. Completion of these tests verify the scalability requirements
defined in section 3.2.3 of \textit{smrt}: Requirements.

\testitem{Directory Scalability}
  {
    This test verifies \textit{smrt}'s ability to handle large quantities of
    directories. The Requirements document defines this as at least 300 albums.
  }
  {
    Navigate to a cityscape menu containing at least 300 directories. Move
    around the cityscape using the up, down, left, and right keys. Select a
    directory. Press escape to return to the cityscape. Press escape to leave
    the cityscape.
  }
  {
    The performance of the system should not be impacted significantly due to
    the size of the menu. There should be no noticeable difference between the
    behavior in this test and the Cityscape Menu Interaction test.
  }
  {
    Performance is slow but usable as sub-levels are loaded.  After that, speed
    is fine (with a capable graphics card).
  }
  {May 5 2006}{David Trowbridge}{pass}

\testitem{Empty Directory Scalability}
  {
    This test verifies \textit{smrt}'s ability to handle directories that
	contain no files or directories.
  }
  {
    Navigate to a cityscape menu containing no files or directories.
  }
  {
    The cityscape menu should display a small table with no items on it.
    All navigation should be disabled except for the back button.
  }
  {}{May 5 2006}{David Trowbridge}{pass}

\testitem{File Scalability}
  {
    This test verifies \textit{smrt}'s ability to manage large quantities of
    files. The Requirements document defines this as at least 2,000 songs, or
    200 movies.
  }
  {
    Navigate to a cityscape menu containing at least 2,000 music files. Move
    around the cityscape using the up, down, left, and right keys. Select a file
    for playback. Press escape to return to the menu. Press escape to leave the
    menu.
  }
  {
    The performance of the system should not be impacted significantly due to
    the size of the menu. There should be no noticeable difference between the
    behavior in this test and the Cityscape Menu Interaction test.

  }
  {
    Like directories, loading takes a while due to I/O.  After this, interaction
    is fine.
  }
  {May 5 2006}{David Trowbridge}{pass}

\testitem{User Interaction Scalability}
  {
    This test demonstrates that \textit{smrt} improves user interaction compared
    to existing media center applications.
  }
  {
    In \textit{smrt}, navigate to a cityscape menu containing video or music
    files. Browse to a predetermined file and activate it by pressing enter.
    Note the number of ``clicks'' required to accomplish this.  Exit
    \textit{smrt}. Start a different media center application such as Freevo.
    Navigate to the same file in Freevo and activate it. Note the number of
    ``clicks'' required to accomplish this.
  }
  {
    The number of button presses required to get to and activate the file in
    \textit{smrt} should be less than the number of button presses in Freevo.
  }
  {
    Tested using freevo and several different files.  Number of clicks was
    smaller, especially in large directories.  In very small directories, the
    difference was not very noticible.
  }
  {May 5 2006}{David Trowbridge}{pass}

\pagebreak

\section{Summary}
The tests defined in this document will be used to verify that the requirements
defined in \textit{smrt}: Requirements have been met. The tests defined here
fall into three categories that reflect the three categories in the requirements
document: display requirements, user interaction requirements, and scalability
requirements. These tests should provide sufficient information to determine
whether \textit{smrt} meets the requirements defined in the requirements
document or not.

\appendix

\section{Related Documents}

\begin{list}{}{
\setlength{\parsep}{1ex}
\setlength{\leftmargin}{0.5in}
\setlength{\itemindent}{-0.5in}
}

\item[] \textbf{[Maccarrone et al. 05]}

	Maccarrone, Cory, Daniel Seikaly, Evan Sheehan, and David Trowbridge.
	\textit{smrt: Requirements}. 2005.

	The requirements document for the \textit{smrt} project.

\item[] \textbf{[Project Looking Glass]}

	http://lg3d-core.dev.java.net/

	The open source web site for Sun's Project Looking Glass.
\end{list}

\end{document}
